\begin{frame}{Indirection}
	\begin{quotation}
		Any problem in computer science can be solved by another layer of indirection.
	\end{quotation}
	\vspace*{-1em}
	\begin{flushright}
		\textup{
		--- David Wheeler
		}
	\end{flushright}
\end{frame}

\begin{frame}{Column Displacements}
	\begin{columns}
		\begin{column}[T]{0.6\textwidth}
			\begin{itemize}[<+->]
				\itemspacing{20pt}
%				\item \makebox[\linewidth][l]{Indirection: Column Displacements}
				\item Indirection: Column Displacements
				\item First displace columns, then displace rows
				\item Fill empty cells with 0
				\item Better compression makes up for more rows
				\item Why it works: fewer nonulls per row
			\end{itemize}
		\end{column}
		
		\begin{column}[T]{0.4\textwidth}
			\centering
			\only<2> {
				\resizebox{0.9\textwidth}{!}{
					\begin{tikzpicture}[align=center, node distance=6cm]
						\matrix (cd) [
							matrix of nodes,
							draw=none,
							row sep = 0.7em,
							column sep = 1.2em,
							ampersand replacement=\&
						] {
							0                \& $\phantom{0}$    \& 0                \& $\phantom{0}$    \& $\s$          \\
							$\s$             \& $\phantom{0}$    \& 0                \& $\phantom{0}$    \& 0             \\
							$\s$             \& $\phantom{0}$    \& $\s$             \& $\phantom{0}$    \& 0             \\
							0                \& $\phantom{0}$    \& 0                \& $\phantom{0}$    \& 0             \\
							$\s$             \& $\phantom{0}$    \& 0                \& $\phantom{0}$    \& $\s$          \\
							$\phantom{0}$    \& $\s$             \& $\phantom{0}$    \& $\phantom{0}$    \& $\phantom{0}$ \\
							$\phantom{0}$    \& $\s$             \& $\phantom{0}$    \& 0                \& $\phantom{0}$ \\
							$\phantom{0}$    \& 0                \& $\phantom{0}$    \& $\s$             \& $\phantom{0}$ \\
							$\phantom{0}$    \& 0                \& $\phantom{0}$    \& 0                \& $\phantom{0}$ \\
							$\phantom{0}$    \& 0                \& $\phantom{0}$    \& $\s$             \& $\phantom{0}$ \\
							$\phantom{0}$    \& $\phantom{0}$    \& $\phantom{0}$    \& 0                \& $\phantom{0}$ \\
						};
						
						\visible<1> {
							\node[draw, fit = (cd-1-1) (cd-11-5), inner sep = 0.9em] (cd-fit) {};
						}
						
						\node[draw, fit = (cd-1-1) (cd-5-1)]  (c1) {};
						\node[draw, fit = (cd-1-3) (cd-5-3)]  (c3) {};
						\node[draw, fit = (cd-1-5) (cd-5-5)]  (c5) {};
						\node[draw, fit = (cd-6-2) (cd-10-2)] (c2) {};
						\node[draw, fit = (cd-7-4) (cd-11-4)] (c4) {};
					\end{tikzpicture}
				}
			}
			\only<3-> {
				\resizebox{0.9\textwidth}{!}{
					\begin{tikzpicture}[align=center, node distance=6cm]
						\matrix (cd) [
							matrix of nodes,
							draw=none,
							row sep = 0.7em,
							column sep = 1.2em,
							ampersand replacement=\&
						] {
							0    \& 0    \& 0    \& 0    \& $\s$ \\
							$\s$ \& 0    \& 0    \& 0    \& 0    \\
							$\s$ \& 0    \& $\s$ \& 0    \& 0    \\
							0    \& 0    \& 0    \& 0    \& 0    \\
							$\s$ \& 0    \& 0    \& 0    \& $\s$ \\
							0    \& $\s$ \& 0    \& 0    \& 0    \\
							0    \& $\s$ \& 0    \& 0    \& 0    \\
							0    \& 0    \& 0    \& $\s$ \& 0    \\
							0    \& 0    \& 0    \& 0    \& 0    \\
							0    \& 0    \& 0    \& $\s$ \& 0    \\
							0    \& 0    \& 0    \& 0    \& 0    \\
						};
						
						\node[draw, fit = (cd-1-1) (cd-11-5), inner sep = 0.9em] (cd-fit) {};
						
						\node[draw, fit = (cd-1-1) (cd-5-1)]  (c1) {};
						\node[draw, fit = (cd-1-3) (cd-5-3)]  (c3) {};
						\node[draw, fit = (cd-1-5) (cd-5-5)]  (c5) {};
						\node[draw, fit = (cd-6-2) (cd-10-2)] (c2) {};
						\node[draw, fit = (cd-7-4) (cd-11-4)] (c4) {};
					\end{tikzpicture}
				}
			}
		\end{column}
	\end{columns}
\end{frame}

\begin{frame}{Column Displacements --- Analysis}
	\begin{itemize}[<+->]
		\itemspacing{20pt}
		\item $m$ columns displacements to store
		\item $\O{n \log \log n + m}$ rows after displacements
		\item $\O{n}$ of them have nonnull row displacement
		\item Potential for more compression
	\end{itemize}
\end{frame}