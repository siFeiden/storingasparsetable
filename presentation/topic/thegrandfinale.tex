\begin{frame}{Putting Together the Pieces}
	\resizebox{\textwidth}{!}{%
		\begin{tikzpicture}[align=center, node distance=2cm]
			\trienode{30}{};
			
			\trienode{33}{below of=n30};
			\draw[edge] (n30-1-3.south) to[out=270, in=90] (n33-1-1.north);
		
			\trienode{24}{left =2cm of n33};
			\draw[edge] (n30-1-2.south) to[out=250, in=70] (n24-1-1.north);
		
			\trienode{19}{right=2cm of n33};
			\draw[edge] (n30-1-5.south) to[out=270, in=90] (n19-1-1.north);
		
			\trienode{47}{below of=n19};
			\draw[edge] (n19-1-4.south) to[out=270, in=90] (n47-1-1.north);
		\end{tikzpicture}
	}
\end{frame}
	
\begin{frame}{Putting Together the Pieces}
	\begin{itemize}[<+->]
		\itemspacing{20pt}
		\item Consider pointers as a table
		\item Compress this table as seen above
		\item In general: use trie with degree $n$
		\item Storage $\O{n}$ and access time $\O{\log_n N}$
		\item Special case: access time $\O{1}$ if $N \in \O{n^c}$
	\end{itemize}
\end{frame}