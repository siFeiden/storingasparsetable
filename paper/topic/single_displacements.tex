In this and the next section, we will show how a static table containing null entries can be compressed such that almost no null entries must be stored.
The method works only for the static case, which means that all data is added to the table before any of it is looked up.
All tables are basically two-dimensional arrays that have the same number of rows and columns, i.e.\ they are square.

\begin{wrapfigure}{r}{0.5\textwidth}
  \begin{tikzpicture}[align=center, node distance=2cm]
%\matrix (all) [table, text width=7mm]
%{
%	0 & 0 & * & 0 \\
%	* & * & * & 0 \\
%	0 & * & * & 0 \\
%	* & 0 & 0 & * \\
%};

%\matrix (r2) [table, text width=7mm, below of=all]
%	{ * & * & * & 0 \\};
%	
%\matrix (r3) [table, text width=7mm, below of=r2-1-3]
%	{ 0 & * & * & 0 \\};
%	
%\matrix (r4) [table, text width=7mm, below of=r3-1-4]
%	{ * & 0 & 0 & * \\};
%	
%\matrix (r1) [table, text width=7mm, below of=r4-1-3]
%	{ 0 & 0 & * & 0 \\};


\matrix (disp) [table, text width=7mm]
{
	2  & *  & *  & *  & 0  & {} & {} & {} & {} & {} & $r(2) = 0$ \\
	3  & {} & {} & 0  & *  & *  & 0  & {} & {} & {} & $r(3) = 2$ \\
	4  & {} & {} & {} & {} & {} & *  & 0  & 0  & *  & $r(4) = 5$ \\
	1  & {} & {} & {} & {} & 0  & 0  & *  & 0  & {} & $r(1) = 3$ \\
	{} & 5  & 6  & 7  & 10 & 11 & 13 & 3  & 0  & 16 & {} \\
};

\end{tikzpicture}
	\caption{ \small Rows of a table displaced and collapsed into a one-dimensional array. \label{fig:row_disp}}
\end{wrapfigure}

The remaining setup is similar to tries:
We store $n$ integers in the range $0$ through $N - 1$ in the tables.
Since all integers in the range must fit in a table, $N$ is the table size.
The tables are square, so $N$ must be a square and there is an $m$ such that $N = m^2$.
$m$ is then the number of rows and columns.
Further we assume that $n \geq \max(2, m)$.
The cell $(i, j)$ for an integer $k$ can be found by counting rowwise from the top left until $k$ or by the following formula: $i = \l\lfloor \frac{k}{m} \r\rfloor + 1$ and $j = \l( k \mod m \r) + 1$.
The table entry for $k$ contains the information belonging to $k$ if $k$ is present in the table and null otherwise.
The compression of such a table $A$ happens in two steps.
Since we consider the static case of table storage, we know all entries before any lookup is performed and we can first insert all $n$ entries into $A$.
Now, we transform $A$ into a one-dimensional array $C$ by mapping every position in $A$ to an index of $C$.
The idea is that we collapse the columns of $A$ into a single nonnull value and thus reduce the number of null entries that need to be stored.
Since a column can have more than one nonnull entry, this is in general not possible for the table $A$.
Instead, we assign a horizontal displacement $r_i$ to every row $i \in \{ 1, \ldots, m \}$.
Such a displacement $r_i$ shifts all entries of row $i$ to the right, independent of the other rows.
If the displacements are well chosen, there will be no columns with more than one nonnull entry and the resulting table can be collapsed into an array $C$.
See Figure~\ref{fig:row_disp} for a graphical example.
Note that the displacement for each row is not calculated in the order in which they appear in the original table.
The reason is the following:
Our goal is to minimize the size of the collapsed table.
However, finding the optimal row displacements for a given table is an NP-complete problem.
Instead, we use a heuristic that gives good results for common use cases.
The rows are first sorted by the number of zeroes they contain.
The new first row is assigned a displacement of 0.
For each remaining row, its displacement is increased stepwise, starting at 0, until no conflict with a previous row occurs.
A conflict are two nonzero entries in the same column.

Assume a table $A$ collapsed into an array $C$ using the row displacements $r_1, \ldots, r_m$.
To look up a number $k$, we first calculate its row $i$ and column $j$ by the formulas from above.
Then, we check if $C( r_i + j )$ contains $k$.
If so, we know that $k$ was present in $A$ and otherwise not.
