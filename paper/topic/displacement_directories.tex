The storage scheme using row and column displacements does a good job at compressing tables.
But still, there is one more improvement that can be made.
Assume again that we transform a table $A$ using the column displacements $c_i$ into a table $B$ and in turn collapse $B$ into an array $C$ using the row displacements $r_i$.
%
\begin{wrapfigure}{r}{0.5\textwidth}
	\centering
	\newcommand{\trienode}[2]{% node label, options for matrix
	\matrix (n#1) [table, text width=7mm, ampersand replacement=\&, #2]
	{ #1 \& {} \& {} \& {} \& {} \\};
	\node[font=\tiny, anchor=south] at (n#1-1-2.north) {0};
	\node[font=\tiny, anchor=south] at (n#1-1-3.north) {1};
	\node[font=\tiny, anchor=south] at (n#1-1-4.north) {2};
	\node[font=\tiny, anchor=south] at (n#1-1-5.north) {3}
}

\begin{tikzpicture}[align=center, node distance=0.5cm and 1cm]
	
	\matrix (ri) [%
			matrix of nodes,
			draw = none,
			row sep = 0.7em,
			column sep = 0em
	]
	{
		0    \\ 0    \\ $\s$ \\
		$\s$ \\ 0    \\ $\s$ \\
		0    \\ $\s$ \\ 0 \\
		0    \\ 0    \\ 0 \\
	};
	\node[draw, fit = (ri-1-1) (ri-12-1)] (ri-fit) {};
	\node[above = of ri-1-1] (rilabel) {$r_i$};
	
	\foreach \n in {1,...,12} {
		\node[font=\scriptsize, left = 0.5cm of ri-\n-1 ] {\n};
	}
	
	
	\matrix (R) [%
			matrix of nodes,
			draw = none,
			row sep = 0.7em,
			column sep = 0em,
			right = of ri
	]
	{
		$r_3$ \\ $r_4$ \\ $r_6$ \\ $r_8$ \\
	};
	\node[draw, fit = (R-1-1) (R-4-1)] (r-fit) {};
	\node[above = of R-1-1] (Rlabel) {$R$};
	
	\foreach \n in {1,...,4} {
		\node[font=\scriptsize, left = 0.25cm of R-\n-1 ] {\n};
	}
	
	
	\matrix (D) [%
			matrix of nodes,
			draw = none,
			row sep = 0.7em,
			column sep = 0em,
			right = of R
	]
	{
		0 \\ 1 \\ 3 \\ 4 \\
	};
	\node[draw, fit = (D-1-1) (D-4-1)] (D-fit) {};
	\node[above = of D-1-1] (Dlabel) {$D$};
	
	
	\matrix (I) [%
			matrix of nodes,
			draw = none,
			row sep = 0.7em,
			column sep = 0em,
			right = of D
	]
	{
		0 \\ 0 \\ 1 \\
		1 \\ 0 \\ 2 \\
		0 \\ 1 \\ 0 \\
		0 \\ 0 \\ 0 \\
	};
	\node[draw, fit = (I-1-1) (I-12-1)] (i-fit) {};
	\node[above = of I-1-1] (Ilabel) {$I$};
	
	
	
	\node[draw, fit = (I-4-1) (I-6-1)] (a) {};
	\node[draw, fit = (I-10-1) (I-12-1)] (b) {};
	
	\node[draw, fit = (ri-4-1) (ri-6-1)] (c) {};
	\node[draw, fit = (ri-10-1) (ri-12-1)] (d) {};
	
\end{tikzpicture}
	\caption{\small Row displacement directory for $p = 12, n = 4$, $d = 3$ with the arrays $R, D$ and $I$. $D$ indexes into $R$, $I$ contains offsets for the indices in $D$. \label{fig:rdi_tables}}
\end{wrapfigure}
%
After the transformation and collapse, we can free the memory of $A$ and $B$ because their information is now stored in $C$.
However, in order to perform a lookup of a name in $C$, we need to know the $r_i$ and $c_i$ as we have seen above.
Let us explore how many row and column displacements must be stored.
The size of $A$ is $m \times m$, so we the number of column displacements we have to store for $A$ is $m$.
The transformation into $B$ increases the number of rows respective to $A$, requiring us to store more than $m$ row displacements.
By an analysis of the double displacement scheme, we find out that there are
$p := \l\lfloor 4 n \log_2 \log_2 n + m + 9.5 n \r\rfloor$ row displacements to be stored and only at most $n$ of them are nonzero.
This means that there is some potential to compress the table for the row displacements which we will examine in the following.

First of all, we split the $p$ entries for the row displacements into $n$ sections.
Each section then has the size $d = p/n = \l\lceil 4 \log_2 \log_2 n + m/n + 9.5 \r\rceil$.
Then, we introduce three new arrays $R$, $D$ and $I$.
$R$ is of size $n$ and contains only the nonzero $r_i$ in ascending order on $i$.
$R(1)$ contains the first nonnull row displacement, $R(2)$ the second one and so on.
The entries in $D$ are indices into $R$.
%$D(j)$ has value $i$ such that $i$ is the largest index of a nonnull row displacement that occurs in the first $j - 1$ sections.
To calculate $D(j)$, we find $i$ such that it is the largest index of a nonnull row displacement occurring in the first $j - 1$ sections.
$D(j)$ is then the index of $r_i$ in $R$.
If no such index $i$ exists, set $D(j) = 0$.
Consider Figure~\ref{fig:rdi_tables}.
To find $D(3)$, we look at the first $3 - 1 = 2$ sections of the row displacement table in the left column.
They are the first six entries, containing two nonnull row displacements.
The largest index of a nonnull row displacement in these six entries is $5$.
$r_5$ occurs in $R$ at position $2$, so $D(3) = 2$.
If we want to calculate $D(1)$, we have to look in the first $0$ sections.
The condition from above cannot be fulfilled because we consider 0 entries and instead we set $D(1) = 0$.
$D(2)$ is similar, since in the first section there is no nonnull row displacement.
Here as well, we set the entry in $D$ to zero.

$I$ has length $nd$, i.e. as many entries as there are row displacements.
The entries in $I$ are offsets to the positions stored in $D$ and they are calculated sectionwise.
For every index $i$ from lower to higher in a section, we store zero at $I(i)$ if $r_i$ is zero.
Otherwise, we first increment a counter which starts at zero and store its value at $I(i)$.
This means that in every section, the entry corresponding to the first nonnull row displacement is set to 1, the second to 2 and so on.
All other entries in a section are 0.

We have split up the list of row displacements into three arrays.
If we want to look up the displacement $r_i$, we first calculate the section $s$ that entry $i$ is in: $s = \l\lceil i / d \r\rceil$.
Then $r_i = R(D(s) + I(i))$.