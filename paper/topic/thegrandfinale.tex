%Having introduced tries and the displacement storage scheme, we can now combine both.
Recall that $N$ is the greatest name that we want to store in our table and $n$ is the number of entries we want to store.
In the beginning, we required that $N$ is bounded by a polynomial of $n$ for both the tries and the displacement scheme.
To overcome this restriction and provide a storage scheme that works well for arbitrary values of $n$ and $N$, we can combine tries and the displacement scheme.
Since we still consider the static case of storage and lookup, we know which values we have to store.
The first step is to insert all these values into a trie with branching factor $n$.
The trie then contains $n$ nodes with $n$ pointers each.
Note that of these $n^2$ pointers, exactly $n - 1$ are nonnull, since every node except the root is pointed at exactly once.
The pointer arrays of each node can be combined into a table of size $n \times n$ and compressed using double displacements (with $m = n$) and a displacement directory.
The properties of the trie data structure combined with the compression methods finally give us a method with storage in $\O{n}$ and access time in $\O{\log_n N}$.
