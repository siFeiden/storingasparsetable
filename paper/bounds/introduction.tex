In this work, we elaborate and explain the techniques presented in \cite{tarjan:storing_sparse_table} to store a map from words to objects with $\O{1}$ lookup and minimised storage needs.
The words are natural numbers in the range $0$ through $N-1$ while the objects can be arbitrary data.
We consider the \emph{static case} where $n$ words are added to the map before any lookups are performed.
In order to apply several optimisations to our storage scheme, we assume that $N$ is bounded by a polynomial in $n$, i.e.~$N \in \O{n^c}$.
This guarantees that the tables we will compress are sparse, but not too sparse.
Use cases are LR parsing and sparse Gaussian elimination which can be implemented using such tables.
In both cases, $c = 2$ giving $N \in \O{n^2}$. \\
First of all, we will introduce \emph{tries} which are a data structure based on trees.
Then, seemingly unrelated, we will present a way to compress tables in three steps using row and column displacements as well as a special indexing scheme.
Finally, we combine both ideas to obtain a way to store a sparse table with good space complexity.